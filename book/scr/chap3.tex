% \documentclass{amsart}
% \usepackage{ctex}
% \usepackage{wyz}

% \usetikzlibrary{decorations.markings}
% \tikzset
%  {every pin/.style = {pin edge = {<-}},
%   > = stealth,
%   flow/.style =
%    {decoration = {markings, mark=at position #1 with {\arrow{>}}},
%     postaction = {decorate}
%    },
%   flow/.default = 0.5,
%   main/.style = {line width=1pt}
%  }

% \newcommand\newtemplate[4][0.18]%
%  {\newsavebox#2%
%   \savebox#2%
%    {\begin{tabular}{@{}c@{}}
%       \begin{tikzpicture}[scale=#1]
%       #4
%       \end{tikzpicture}\\[-1ex]
%       \templatecaption{#3}\\[-1ex]
%     \end{tabular}%
%    }%
%  }
% \newcommand\template[1]{\usebox{#1}}
% \newcommand\templatecaption[1]{{\sffamily\scriptsize#1}}
% \newcommand\Tr{\mathop{\mathrm{Tr}}}

% \newtemplate\sink{sink}%
%  {\foreach \sx in {+,-}                   %
%    {\draw[flow] (\sx4,0) -- (0,0);        %
%     \draw[flow] (0,\sx4) -- (0,0);        %
%     \foreach \sy in {+,-}                 %
%       \foreach \a/\b in {2/1,3/0.44}      %
%         \draw[flow,domain=\sx\a:0] plot (\x, {\sy\b*\x*\x});
%    }
%  }

% \newtemplate\source{source}%
%  {\foreach \sx in {+,-}                   %
%    {\draw[flow] (0,0) -- (\sx4,0);        %
%     \draw[flow] (0,0) -- (0,\sx4);        %
%     \foreach \sy in {+,-}                 %
%       \foreach \a/\b in {2/1,3/0.44}      %
%         \draw[flow,domain=0:\sx\a] plot (\x, {\sy\b*\x*\x});
%    }
%  }

% \newtemplate\stablefp{line of stable fixed points}%
%  {\draw (-4,0) -- (4,0);                  %
%   \foreach \sy in {+,-}                   %
%    {\draw[flow] (0,\sy4) -- (0,0);        %
%     \foreach \x in {-3,-2,-1,1,2,3}       %
%       \draw[flow] (\x,\sy3) -- (\x,0);
%    }
%  }

% \newtemplate\unstablefp{line of unstable fixed points}%
%  {\draw (-4,0) -- (4,0);                  %
%   \foreach \sy in {+,-}                   %
%    {\draw[flow] (0,0) -- (0,\sy4);        %
%     \foreach \x in {-3,-2,-1,1,2,3}       %
%       \draw[flow] (\x,0) -- (\x,\sy3);
%    }
%  }

% \newtemplate\spiralsink{spiral sink}%
%  {\draw (-4,0) -> (4,0);                  %
%   \draw (0,-4) -- (0,4);                  %
%   \draw [samples=100,smooth,domain=27:7]  %
%        plot ({\x r}:{0.005*\x*\x});       %
%   \def\x{26}                              %
%   \draw[->] ({\x r}:{0.005*\x*\x}) -- +(0.01,-0.01);%
%  }

% \newtemplate\spiralsource{spiral source}%
%  {\draw (-4,0) -> (4,0);                  %
%   \draw (0,-4) -> (0,4);                  %
%   \draw [samples=100,smooth,domain=10:28] %
%        plot ({-\x r}:{0.005*\x*\x});      %
%   \def\x{27.5}                            %
%   \draw[<-] ({-\x r}:{0.005*\x*\x}) -- +(0.01,-0.008);%
%  }

% \newtemplate[0.15]\centre{center}%
%  {\draw (-5,0) -- (4,0);                  %
%   \draw (0,-4) -- (0,4);                  %
%   \foreach \r in {1,2,3}                  %
%     \draw[flow=0.63] (\r,0) arc (0:-360:\r cm);
%  }

% \newtemplate\saddle{saddle}%
%  {\foreach \sx in {+,-}                   %
%    {\draw[flow] (\sx4,0) -- (0,0);        %
%     \draw[flow] (0,0) -- (0,\sx4);        %
%     \foreach \sy in {+,-}                 %
%       \foreach \a/\b/\c/\d in {2.8/0.3/0.7/0.6, 3.9/0.4/1.3/1.1}
%         \draw[flow] (\sx\a,\sy\b)         %
%           .. controls (\sx\c,\sy\d) and (\sx\d,\sy\c)
%           .. (\sx\b,\sy\a);
%    }
%  }

% \newtemplate\degensink{degenerate sink}%
%  {\draw (0,-4) -- (0,4);                  %
%   \foreach \s in {+,-}                    %
%    {\draw[flow] (\s4,0) -- (0,0);         %
%     \foreach \a/\b/\c/\d in {3.5/4/1.5/1, 2.5/2/1/0.8}
%       \draw[flow] (\s-3.5,\s\a)           %
%         .. controls (\s\b,\s\c) and (\s\b,\s\d)
%         .. (0,0);
%    }
%  }

% \newtemplate\degensource{degenerate source}%
%  {\draw (0,-4) -- (0,4);                  %
%   \foreach \s in {+,-}                    %
%    {\draw[flow] (0,0) -- (\s4,0);         %
%     \foreach \a/\b/\c/\d in {3.5/4/1.5/1, 2.5/2/1/0.8}
%       \draw[flow] (0,0)                   %
%         .. controls (\s\b,\s\d) and (\s\b,\s\c)
%         .. (\s-3.5,\s\a);
%    }
%    }

% \title{细焦点}
% \author{}
% \date{}

% % \begin{document}
% \maketitle
\part{细焦点}
\section{微分方程的双曲平衡点}
\subsection{线性方程的平衡点}
\subsection{非线性方程的平衡点}

\section{后继函数法}
\subsection{后继函数法}
\begin{lemma}
  \label{lemma1}

\end{lemma}

\begin{proof}[ZYJ]
如果
\end{proof}

\begin{proof}[YSY]
如果
\end{proof}

\begin{proof}[WYZ]
  首先考虑一个函数$h(x)$是周期函数,
  我们找它的原函数还是周期函数的条件.
  如果$h(x+l)=h(x)$,
  那么
\end{proof}

\begin{defination}[解析函数(复变函数)]\ref{1}
  如果函数$w=f(z)$在区域$D$内可微,则称$f(z)$为区域$D$内的 \textbf{解析函数},
  或称$f(z)$在区域D内解析.
\end{defination}

\begin{defination}[解析函数(实变函数)]\ref{2}
如果函数$y=f(x)$在区间$D$内可以展开成Taylor级数,则称$f(x)$为区间$D$上的\textbf{解析函数}.
\end{defination}

\begin{note}[反例1]
  Taylor展开存在但是不收敛到原函数.
\end{note}
\subsubsection{焦点量}

\begin{note}
  焦点量,鞍点量与奇点量.
\end{note}
\subsubsection{例子}
\subsubsection{技巧:常见的三角为0的积分}
\begin{enumerate}
\item 三角函数的正交性
\(1,\sin(x),\cos(x),\sin(2x),\cos(2x),\sin(3x),\cos(3x),\dots\)\\
它们中的任意两个不同的函数的乘积在\([0,2\pi]\)上的积分等于0.
\[\int_{0}^{2\pi} 1\mathbf{d}x=2\pi\]
\[\int_{0}^{2\pi} \cos^2(nx) \mathbf{d}x=\pi\]
\[\int_{0}^{2\pi} \sin^2(nx) \mathbf{d}x=\pi\]
\item\( \sin^n(x)\cos^m(x)\)
  \begin{enumerate}
  \item 有一个是奇数
\[\int_{0}^{2\pi} \sin^n(x) \mathbf{d}x=\int_{-\pi}^{\pi} \sin^n(x) \mathbf{d}x=0\]
,n是奇数。\\
\[\int_{0}^{2\pi} \cos^n(x) \mathbf{d}x=
  \int_{\frac{\pi}{2}}^{-\frac{3\pi}{2}} \cos^n(\frac{\pi}{2}-x) \mathbf{d}(\frac{\pi}{2}-x)=
   \int_{\frac{\pi}{2}}^{-\frac{3\pi}{2}} \sin^n(x) \mathbf{d}x=
   - \int_{0}^{2\pi} \sin^n(x) \mathbf{d}x=
  0\]
,n是奇数。\\
\[\int_{0}^{2\pi}\sin^n(x) \cos^m(x) \mathbf{d}x=0\],如果m和n中有一个是奇数.\\
\item 都是偶数
如果m和n都是偶数,
就用公式
\[\sin ^2 x=\frac{1-\cos 2x}{2},\cos ^2 x =\frac{1+\cos 2x}{2}\].
,一般不等于0.
\end{enumerate}
\item 幂次的积分
  \[\int_{0}^{\frac{\pi}{2}}\sin^n(x) \mathbf{d}x
=\int_{0}^{\frac{\pi}{2}}\cos^n(x) \mathbf{d}x
=
\left\{
\begin{aligned}
&\frac{n-1}{n} \times \frac{n-3}{n-2} \times \dots \frac{4}{5} \times \frac{2}{3},\text{n是奇数}\\
&\frac{n-1}{n} \times \frac{n-3}{n-2} \times \dots \frac{3}{4} \times \frac {1}{2} \times \frac{n}{2},\text{n是偶数}
\end{aligned}
\right.\]
\end{enumerate}

\subsection{后继函数法的改进}
我们不难发现,在计算焦点量的过程中,影响计算速度的主要有两个
一个是$(0,\theta)$的定积分,
一个是$(0,\pi)$的不定积分。
在用计算机代数系统来计算的时候,
会花费掉大量的时间,
所以在遇到的问题比价复杂,计算量很大的时候,
应该找一种方法,
来避开这里的定积分和不定积分,
节约计算资源,节约计算的时间成本。
有一种方法是,
把计算机不擅长处理的积分问题化成计算机擅长处理的迭代和求留数。
\subsubsection{一种迭代求焦点量的方法}

\subsubsection{留数与积分}
\begin{defination} [留数]

\end{defination}
\section{形式级数法}
\subsection{Lyapunov稳定性}
粗略的讲,
所谓Lyapunov稳定性,
对一个微分方程的特解,
就是当初始值发生微小的扰动,
它的解的变化还是在微小的范围内,
我们就说它是Lyapunov稳定的.
\subsection{Lyapunov判别法}

\subsection{形式级数}
一般来讲,没有证明收敛性的级数,我们都叫它形式级数.
\subsection{形式级数法}

\begin{description}
\item[Step1] 我们假设形式级数为

\item[Step2] 我们将它带入到方程里面就有

\item[Step3] 依次求出形式级数的前面的几项

\end{description}
\section{对称性}
\subsection{可返性}

\begin{description}
\item[对称系统]
\item[可返系统]
\end{description}
\subsection{对称性}

\section{直接利用周期的方法}
\subsection{直接利用}

\section{二次系统以后结果}
除了线性系统,二次系统是最平凡最简单的平面微分方程的系统,
然而即使是二次系统,
任然有很多问题没有解决.
下面给出一些结果.


\begin{theorem}

\end{theorem}

\section{三次系统以后结果}
\begin{theorem}

\end{theorem}

\section{多项式的理想与Singular入门}

\end{document}